\documentclass[12pt]{article}
\usepackage{pmmeta}
\pmcanonicalname{SpectralRadius}
\pmcreated{2013-03-22 13:13:58}
\pmmodified{2013-03-22 13:13:58}
\pmowner{Koro}{127}
\pmmodifier{Koro}{127}
\pmtitle{spectral radius}
\pmrecord{11}{33703}
\pmprivacy{1}
\pmauthor{Koro}{127}
\pmtype{Definition}
\pmcomment{trigger rebuild}
\pmclassification{msc}{58C40}
\pmdefines{spectrum}

% this is the default PlanetMath preamble.  as your knowledge
% of TeX increases, you will probably want to edit this, but
% it should be fine as is for beginners.

% almost certainly you want these
\usepackage{amssymb}
\usepackage{amsmath}
\usepackage{amsfonts}

% used for TeXing text within eps files
%\usepackage{psfrag}
% need this for including graphics (\includegraphics)
%\usepackage{graphicx}
% for neatly defining theorems and propositions
%\usepackage{amsthm}
% making logically defined graphics
%%%\usepackage{xypic}

% there are many more packages, add them here as you need them

% define commands here
\begin{document}
If $V$ is a vector space over $\mathbb{C}$, the spectrum of a linear mapping $T:V\rightarrow V$ is the set
\[\sigma(T) = 
\{\lambda\in \mathbb{C}: T-\lambda I \mbox{is not invertible}\},\]
where $I$ denotes the identity mapping. 
If $V$ is finite dimensional, the spectrum of $T$ is precisely the set of its eigenvalues. For infinite dimensional spaces this is not generally true,
although it is true that each eigenvalue of $T$ belongs to $\sigma(T)$. The \emph{spectral radius} of $T$ is 
\[\rho(T) = \sup \{|\lambda|:\lambda\in\sigma(T)\}.\]

More generally, the spectrum and spectral radius can be defined for Banach algebras with identity element: If $\mathcal{A}$ is a Banach algebra over $\mathbb{C}$ with identity element $e$, the spectrum of an element $a \in \mathcal{A}$ is the set
 $$\sigma(a) = \{ \lambda \in \mathbb{C} : a - \lambda e \mbox{is not invertible in} \mathcal{A} \}$$

The spectral radius of $a$ is $\rho(a) = \sup \{| \lambda | : \lambda \in \sigma(a) \}$.

%%%%%
%%%%%
\end{document}
