\documentclass[12pt]{article}
\usepackage{pmmeta}
\pmcanonicalname{GeneralizedToposesWithManyvaluedLogicSubobjectClassifiers}
\pmcreated{2013-03-22 18:13:11}
\pmmodified{2013-03-22 18:13:11}
\pmowner{bci1}{20947}
\pmmodifier{bci1}{20947}
\pmtitle{generalized toposes with many-valued logic subobject classifiers}
\pmrecord{60}{40803}
\pmprivacy{1}
\pmauthor{bci1}{20947}
\pmtype{Topic}
\pmcomment{trigger rebuild}
\pmclassification{msc}{58A03}
\pmclassification{msc}{18B25}
\pmclassification{msc}{03B15}
\pmclassification{msc}{03G30}
\pmclassification{msc}{03G20}
\pmclassification{msc}{03B50}
\pmsynonym{LMn-algebraic n-valued logic}{GeneralizedToposesWithManyvaluedLogicSubobjectClassifiers}
\pmsynonym{algebraic category of $LM_n$ logic algebras}{GeneralizedToposesWithManyvaluedLogicSubobjectClassifiers}
%\pmkeywords{generalized topoi with many-valued logic subobject classifiers}
%\pmkeywords{the category of n-valued}
%\pmkeywords{LMn-logic algebras and LMn-lattice morphisms}
%\pmkeywords{n-valued logic algebra}
%\pmkeywords{algebraic catgeory of n-valued logic lattices and lattice-morphisms}
\pmrelated{NonAbelianStructures}
\pmrelated{AbelianCategory}
\pmrelated{AxiomsForAnAbelianCategory}
\pmrelated{GeneralizedVanKampenTheoremsHigherDimensional}
\pmrelated{AxiomaticTheoryOfSupercategories}
\pmrelated{CategoricalOntology}
\pmrelated{NonCommutingGraphOfAGroup}
\pmrelated{NonAbelianStructures}
\pmrelated{QuantumLogicsTopoi}
\pmrelated{CategoricalAlgebr}
\pmdefines{many-valued logic subobject classifier}
\pmdefines{algebraic category of LMn logic algebras}
\pmdefines{noncommutative lattice}

% this is the default PlanetMath preamble.  as your % define commands here
%%\usepackage{xypic}
\usepackage[mathscr]{eucal}
\usepackage{amsmath, amssymb, amsfonts, amsthm, amscd, latexsym,color,enumerate}
%%\usepackage{xypic}
\usepackage[mathscr]{eucal}
\usepackage{setspace}
\xyoption{curve}
\theoremstyle{plain}
\newtheorem{lemma}{Lemma}[section]
\newtheorem{proposition}{Proposition}[section]
\newtheorem{theorem}{Theorem}[section]
\newtheorem{corollary}{Corollary}[section]
\theoremstyle{definition}
\newtheorem{definition}{Definition}[section]
\newtheorem{example}{Example}[section]
%\theoremstyle{remark}
\newtheorem{remark}{Remark}[section]
\newtheorem*{notation}{Notation}
\newtheorem*{claim}{Claim}
\newtheorem{exe}{Example}[section]

\renewcommand{\thefootnote}{\ensuremath{\fnsymbol{footnote%%@
}}} \numberwithin{equation}{section}

\newcommand{\Ad}{{\rm Ad}}
\newcommand{\Aut}{{\rm Aut}}
\newcommand{\Cl}{{\rm Cl}}
\newcommand{\Co}{{\rm Co}}
\newcommand{\DES}{{\rm DES}}
\newcommand{\Diff}{{\rm Diff}}
\newcommand{\Dom}{{\rm Dom}}
\newcommand{\Hol}{{\rm Hol}}
\newcommand{\Mon}{{\rm Mon}}
\newcommand{\Hom}{{\rm Hom}}
\newcommand{\Ker}{{\rm Ker}}
\newcommand{\Ind}{{\rm Ind}}
\newcommand{\IM}{{\rm Im}}
\newcommand{\Is}{{\rm Is}}
\newcommand{\ID}{{\rm id}}
\newcommand{\GL}{{\rm GL}}
\newcommand{\Iso}{{\rm Iso}}
\newcommand{\Sem}{{\rm Sem}}
\newcommand{\St}{{\rm St}}
\newcommand{\Sym}{{\rm Sym}}
\newcommand{\SU}{{\rm SU}}
\newcommand{\Tor}{{\rm Tor}}
\newcommand{\U}{{\rm U}}

\newcommand{\A}{\mathcal A}
\newcommand{\Ce}{\mathcal C}
\newcommand{\D}{\mathcal D}
\newcommand{\E}{\mathcal E}
\newcommand{\F}{\mathcal F}
\newcommand{\G}{\mathcal G}
\newcommand{\Q}{\mathcal Q}
\newcommand{\R}{\mathcal R}
\newcommand{\cS}{\mathcal S}
\newcommand{\cU}{\mathcal U}
\newcommand{\W}{\mathcal W}

\newcommand{\bA}{\mathbb{A}}
\newcommand{\bB}{\mathbb{B}}
\newcommand{\bC}{\mathbb{C}}
\newcommand{\bD}{\mathbb{D}}
\newcommand{\bE}{\mathbb{E}}
\newcommand{\bF}{\mathbb{F}}
\newcommand{\bG}{\mathbb{G}}
\newcommand{\bK}{\mathbb{K}}
\newcommand{\bM}{\mathbb{M}}
\newcommand{\bN}{\mathbb{N}}
\newcommand{\bO}{\mathbb{O}}
\newcommand{\bP}{\mathbb{P}}
\newcommand{\bR}{\mathbb{R}}
\newcommand{\bV}{\mathbb{V}}
\newcommand{\bZ}{\mathbb{Z}}

\newcommand{\bfE}{\mathbf{E}}
\newcommand{\bfX}{\mathbf{X}}
\newcommand{\bfY}{\mathbf{Y}}
\newcommand{\bfZ}{\mathbf{Z}}

\renewcommand{\O}{\Omega}
\renewcommand{\o}{\omega}
\newcommand{\vp}{\varphi}
\newcommand{\vep}{\varepsilon}

\newcommand{\diag}{{\rm diag}}
\newcommand{\grp}{{\mathbb G}}
\newcommand{\dgrp}{{\mathbb D}}
\newcommand{\desp}{{\mathbb D^{\rm{es}}}}
\newcommand{\Geod}{{\rm Geod}}
\newcommand{\geod}{{\rm geod}}
\newcommand{\hgr}{{\mathbb H}}
\newcommand{\mgr}{{\mathbb M}}
\newcommand{\ob}{{\rm Ob}}
\newcommand{\obg}{{\rm Ob(\mathbb G)}}
\newcommand{\obgp}{{\rm Ob(\mathbb G')}}
\newcommand{\obh}{{\rm Ob(\mathbb H)}}
\newcommand{\Osmooth}{{\Omega^{\infty}(X,*)}}
\newcommand{\ghomotop}{{\rho_2^{\square}}}
\newcommand{\gcalp}{{\mathbb G(\mathcal P)}}

\newcommand{\rf}{{R_{\mathcal F}}}
\newcommand{\glob}{{\rm glob}}
\newcommand{\loc}{{\rm loc}}
\newcommand{\TOP}{{\rm TOP}}

\newcommand{\wti}{\widetilde}
\newcommand{\what}{\widehat}

\renewcommand{\a}{\alpha}
\newcommand{\be}{\beta}
\newcommand{\ga}{\gamma}
\newcommand{\Ga}{\Gamma}
\newcommand{\de}{\delta}
\newcommand{\del}{\partial}
\newcommand{\ka}{\kappa}
\newcommand{\si}{\sigma}
\newcommand{\ta}{\tau}
\def \C{\mathsf{C}}
\newcommand{\med}{\medbreak}
\newcommand{\medn}{\medbreak \noindent}
\newcommand{\bign}{\bigbreak \noindent}

%%N-Logic specific macros%%


\newcommand{\B}{{\mathcal B}}

\newcommand{\MV}{{\mathcal MV}}

\newcommand{\LM}{{\mathcal LM}_n}

\newcommand{\CLM}{{\mathcal CLM}_n}

\newcommand{\Post}{{\mathcal P}ost_n}

\newcommand{\Ra}{\Rightarrow}

\newcommand{\GMn}{$NMV_n$}

\newcommand{\LMn}{$LM_n$}

\newcommand{\NMVn}{NMV_n}

\newtheorem{defi}{Definition}

\newtheorem{prop}{Proposition}

\newtheorem{rem}{Remark}

\newtheorem{coro}{Corollary}

\newtheorem{lema}{Lemma}

\newcommand{\lmn}{LM_{n}}

\newcommand{\lmt}{LM_{\Th}}

\newcommand{\ov}{\overline}

\newcommand{\sm}{\smile}

\newcommand{\lmrn}{LMR_{n}}

\renewcommand{\t}{\times}

\newcommand{\eu}{\equiv_{1}}

\newcommand{\lu}{\leq_{1}}

\newcommand{\ez}{\equiv_{0}}

\newcommand{\lz}{\leq_{0}}

\newcommand{\Su}{S_{1}}

\newcommand{\Sd}{S_{2}}

\newcommand{\Si}{S_{i}}

\newcommand{\pri}{pr_{i}}

\newcommand{\pru}{pr_{1}}

\newcommand{\prd}{pr_{2}}

\newcommand{\pii}{\pi_{i}}

\newcommand{\piu}{\pi_{1}}

\newcommand{\pid}{\pi_{2}}

\newcommand{\unudoi}{\{1,2\}}

\newcommand{\Li}{L_{i}}

\newcommand{\Lu}{L_{1}}

\newcommand{\Ld}{L_{2}}

\newcommand{\ioi}{\iota_{i}}

\newcommand{\iou}{\iota_{1}}

\newcommand{\iod}{\iota_{2}}

\newcommand{\iii}{\iou\t\iod}

\newcommand{\K}{{\cal K}}

\newcommand{\sta}{\stackrel}

\newcommand{\fu}{f_{1}}

\newcommand{\fn}{f_{n}}

\newcommand{\fk}{f_{k}}

\newcommand{\e}{\eta}

\newcommand{\eps}{\epsilon}

\newcommand{\SLI}{S^{[I]}}

\newcommand{\CSLI}{C(S^{[I]})}

\newcommand{\CLI}{C(L)^{[I]}}

\newcommand{\mSI}{\models_{\SI}}

\newcommand{\mS}{\models_{\S}}

\newcommand{\tu}{t_{1}}

\newcommand{\tn}{t_{n}}

\newcommand{\tk}{t_{k}}

\renewcommand{\S}{\Sigma}

\newcommand{\s}{\sigma}

\newcommand{\Sn}{\Sigma_{n}}

\newcommand{\Sw}{\Sigma_{w}}

\newcommand{\wu}{w_{1}}

\newcommand{\wn}{w_{n}}

\newcommand{\wi}{w_{i}}

\newcommand{\wj}{w_{j}}

\newcommand{\yu}{y_{1}}

\newcommand{\yn}{y_{n}}

\newcommand{\yi}{y_{i}}

\newcommand{\yj}{y{j}}

\newcommand{\zu}{z_{1}}

\newcommand{\zn}{z_{n}}

\newcommand{\zk}{z_{k}}

\newcommand{\zi}{z_{i}}

\newcommand{\xu}{x_{1}}

\newcommand{\xn}{x_{n}}

\renewcommand{\xi}{x_{i}}

\newcommand{\xj}{x_{j}}

\newcommand{\wk}{w_{k}}

\newcommand{\xk}{x_{k}}

\newcommand{\yk}{y_{k}}

\newcommand{\su}{s_{1}}

\newcommand{\sn}{s_{n}}



\newcommand{\sj}{s_{j}}

\newcommand{\sk}{s_{k}}



\newcommand{\Imp}{\Rightarrow}



\newcommand{\SI}{\S_{I}}

\newcommand{\EI}{E_{I}}

\renewcommand{\phi}{\varphi}



\newcommand{\bw}{\bigwedge}

\newcommand{\bv}{\bigvee}

\newcommand{\w}{\wedge}

\renewcommand{\v}{\vee}



\renewcommand{\phi}{\varphi}

\newcommand{\phii}{\phi_{i}}

\newcommand{\phij}{\phi_{j}}

\newcommand{\phik}{\phi_{k}}

\newcommand{\phiu}{\phi_{1}}

\newcommand{\phin}{\phi_{n-1}}



\newcommand{\unen}{\{1,\ldots,n\}}

\newcommand{\ffu}{f_{1}}

\newcommand{\ffn}{f_{n}}

\newcommand{\ffi}{f_{i}}

\newcommand{\ffj}{f_{j}}

\newcommand{\ffk}{f_{k}}



\newcommand{\orc}{\forall}

\newcommand{\exi}{\exists}



\newcommand{\au}{a_{1}}

\newcommand{\an}{a_{n}}

\newcommand{\ai}{a_{i}}

\newcommand{\aj}{a_{j}}

\newcommand{\ak}{a_{k}}

\newcommand{\bu}{b_{1}}

\newcommand{\bn}{b_{n}}

\newcommand{\bi}{b_{i}}

\newcommand{\bj}{b_{j}}

\newcommand{\bk}{b_{k}}

\newcommand{\ra}{\rightarrow}



\renewcommand{\P}{{\cal P}}

\newcommand{\N}{{I\!\!N}}



\newcommand{\p}{\oplus}

\newcommand{\cd}{\odot}

\newcommand{\unmu}{\{1,\ldots,n-1\}}



\newcommand{\Ss}{S_{\sigma}}

\newcommand{\Th}{\Theta}

\newcommand{\ThS}{\Theta_{\S}}



\newcommand{\Luk}{ {\cal L}uk_{n} }

\newcommand{\Gen}{{\cal NMVA}_{n}}

\newcommand{\CGen}{{\cal NMVA}_{n}}

\newcommand{\GR}{{\cal NLGU}_{n}}
\newcommand{\gr}{NLGU_{n}}

\newcommand{\NMVN}{{\cal NMVA}_n}

\newcommand{\lra}{{\longrightarrow}}

\newcommand{\rat}{{\rightarrowtail}}
\newcommand{\oset}[1]{\overset {#1}{\ra}}
\newcommand{\osetl}[1]{\overset {#1}{\lra}}
\newcommand{\hr}{{\hookrightarrow}}
\newcommand{\labto}[1]{\overset{#1}{\lra}}
\newcommand{\midsq}[1]{\save\go[0,0];[1,1]:(0.5,0) \drop{#1}\restore}
%the following allows for longer arrows with long labels
%the first entry gives the additional length and the second gives the label
%e.g. \llabto{0.75}{\rm{long label}}
\newcommand{\llabto}[2]{\stackrel{#2}
{\rule[0.5ex]{#1 em}{0.05ex}\hspace{-0.4em}\longrightarrow}}

%the next gives two direction arrows at the top of a 2 x 2 matrix

\newcommand{\directs}[2]{\def\objectstyle{\scriptstyle} \objectmargin={0pt}
\xy (0,4)*+{}="a",(0,-2)*+{\rule{0em}{1.5ex}#2}="b",(7,4)*+{\;#1}="c" \ar@{->} "a";"b"
\ar @{->}"a";"c" \endxy }
%the next gives two direction arrows at the middle of a 2 x 2 matrix

\newcommand{\xdirects}[2]{\def\objectstyle{\scriptstyle} \objectmargin={0pt}
\xy (0,0)*+{}="a",(0,-6)*+{\rule{0em}{1.5ex}#2}="b",(7,0)*+{\;#1}="c" \ar@{->} "a";"b"
\ar @{->}"a";"c" \endxy }
%and this is smaller for a 1 x 1 matrix
\newcommand{\sdirects}[2]{\def\objectstyle{\scriptstyle} \objectmargin={0pt}
\xy (0,2.2)*+{}="a",(0,-2.5)*+{\rule{0em}{1.5ex}#2}="b",(7,2.2)*+{\;#1}="c" \ar@{->}
"a";"b" \ar @{->}"a";"c" \endxy }

% the following are codes for the identities and connections
\newcommand{\bl}{\mbox{\rule{0.08em}{1.7ex}\hspace{-0.00em}\rule{0.7em}{0.2ex}}}

\newcommand{\br}{\mbox{\rule{0.7em}{0.2ex}\hspace{-0.04em}\rule{0.08em}{1.7ex}}}

\newcommand{\tr}{\mbox{\rule[1.5ex]{0.7em}{0.2ex}\hspace{-0.03em}\rule{0.08em}{1.7ex}}}

\newcommand{\tl}{\mbox{\rule{0.08em}{1.7ex}\rule[1.54ex]{0.7em}{0.2ex}}}

\newcommand{\hh}{\mbox{\rule{0.7em}{0.2ex}\hspace{-0.7em}\rule[1.5ex]{0.70em}{0.2ex}}}

\newcommand{\vv}{\mbox{\rule{0.08em}{1.7ex}\hspace{0.6em}\rule{0.08em}{1.7ex}}}

\newcommand{\sq}{\mbox{\rule{0.08em}{1.7ex}\hspace{-0.00em}\rule{0.7em}{0.2ex}\hspace{-0.7em}\rule[1.54ex]{0.7em}{0.2ex}\hspace{-0.03em}\rule{0.08em}{1.7ex}}}

\newcommand{\tsq}{\mbox{\rule{0.04em}{1.55ex}\hspace{-0.00em}\rule{0.7em}{0.1ex}\hspace{-0.7em}\rule[1.5ex]{0.7em}{0.1ex}\hspace{-0.03em}\rule{0.04em}{1.55ex}}}
\newcommand{\tssq}{\mbox{\rule{0.04em}{1.55ex}\hspace{-0.00em}\rule{0.7em}{0.1ex}\hspace{-0.7em}\rule[1.5ex]{0.7em}{0.1ex}\hspace{-0.03em}\rule{0.04em}{1.55ex}\hspace{-0.43em}\rule[0.7ex]{0.1em}{0.2ex}}}

\newcommand{\quads}[4]{\left( #1 \hspace{1em}
\overset{\textstyle{#2}}{\underset{\textstyle{#4}} {\rule{0mm}{1mm}}} \hspace{1em} #3
\right)}
\def\bu{\bullet}
\def\prt{\partial}
\def\eps{\varepsilon}
\def\red{\textcolor{red}}
\def\blue{\textcolor{blue}}
\def\leq{\leqslant}
\def\geq{\geqslant}
\def\le{\leqslant}
\def\ge{\geqslant}

\begin{document}
\section{Generalized toposes}
\subsection{Introduction}

 \emph{Generalized topoi (toposes) with many-valued algebraic logic subobject classifiers}
are specified by the associated categories of algebraic logics previously defined as $LM_n$, that is, {\em non-commutative} lattices with $n$ logical values, where $n$ can also be chosen to be any cardinal, including infinity, etc.

\subsection{Algebraic category of $LM_n$ logic algebras}
 
 \L{}ukasiewicz \emph{logic algebras} were constructed by Grigore Moisil in 1941 to define `nuances' in logics, or many-valued logics, as well as 3-state control logic (electronic) circuits. \L{}ukasiewicz-Moisil ($LM_n$) logic algebras were defined axiomatically in 1970, in ref. \cite{GG-CV70}, as n-valued logic algebra representations and extensions of the \L ukasiewcz (3-valued) logics; then, the universal properties of categories of $LM_n$ -logic algebras were also investigated and reported in a series of recent publications (\cite{GG2k6} and references cited therein). Recently, several modifications of {\em $LM_n$-logic algebras} are under consideration as valid candidates for representations of {\em quantum logics}, as well as for modeling non-linear biodynamics in genetic `nets' or networks (\cite{ICB77}), and in single-cell organisms, or in tumor growth. For a recent review on $n$-valued logic algebras, and major published results, the reader is referred to \cite{GG2k6}.

 The \emph{category $\mathcal{LM}$ of \L{}ukasiewicz-Moisil, $n$-valued logic algebras ($LM_n$), and $LM_n$--lattice morphisms}, $\lambda_{LM_n}$, was introduced in 1970 in ref. \cite{GG-CV70} as an algebraic category tool for $n$-valued logic studies. The objects of $\mathcal{LM}$ are the \emph{non--commutative} $LM_n$ lattices and the morphisms of $\mathcal{LM}$ are the $LM_n$-lattice morphisms as defined next.

\begin{definition}\rm

A {\it $n$--valued \L ukasiewicz--Moisil algebra}, ({\it $LM_{n}$--algebra}) is a structure of the form
$(L,\vee,\wedge,N,(\phii)_{i\in\{1,\ldots,n-1\}},0,1)$, subject to the following axioms:
\begin{itemize}
\item (L1) $(L,\vee,\wedge,N,0,1)$ is a {\it de Morgan algebra}, that is, a bounded distributive lattice with a decreasing involution $N$ satisfying the de Morgan property $N({x\vee y})=Nx\wedge Ny$;
\item (L2) For each $i\in\{1,\ldots,n-1\}$, $\phii:L\lra L$ is a lattice endomorphism;\footnote{The $\phii$'s are called the {\em Chrysippian endomorphisms} of $L$.}
\item (L3) For each $i\in\{1,\ldots,n-1\},x\in L$, $\phii(x)\vee N{\phii(x)}=1$ and
$\phii(x)\wedge N{\phii(x)}=0$;
\item (L4) For each $i,j\in\{1,\ldots,n-1\}$, $\phii\circ\phi_{j}=\phi_{k}$ iff $(i+j)= k$;
\item (L5) For each $i,j\in\{1,\ldots,n-1\}$, $i\leq j$ implies $\phii\leq\phi_{j}$;
\item (L6) For each $i\in\{1,\ldots,n-1\}$ and $x\in L$, $\phii(N x)=N\phi_{n-i}(x)$.
\item (L7) Moisil's `determination principle':
$$\left[\orc i\in\{1,\ldots,n-1\},\;\phii(x)=\phii(y)\right] \; implies \; [x = y] \;$$
\cite{GG-CV70,GG2k6}.
\end{itemize}
\end{definition}

\begin{exe}\rm
Let $L_n=\{0,1/(n-1),\ldots,(n-2)/(n-1),1\}$. This set can be naturally endowed with an $\mbox{LM}_n$
--algebra structure as follows:
\begin{itemize}
\item the bounded lattice operations are those induced by the usual order on rational numbers;
\item for each $j\in\{0,\ldots,n-1\}$, $N(j/(n-1))=(n-j)/(n-1)$;
\item for each $i\in\{1,\ldots,n-1\}$ and $j\in\{0,\ldots,n-1\}$,
$\phii(j/(n-1))=0$ if $j<i$ and $=1$ otherwise.
\end{itemize}
\end{exe}
Note that, for $n=2$, $L_n=\{0,1\}$, and there is only one Chrysippian endomorphism of $L_n$ is $\phi_1$, which
is necessarily restricted by the determination principle to a bijection, thus making $L_n$ a Boolean algebra (if
we were also to disregard the redundant bijection $\phi_1$). Hence, the `overloaded' notation $L_2$, which is
used for both the classical Boolean algebra and the two--element $\mbox{LM}_2$--algebra, remains consistent.
\begin{exe}\rm
Consider a Boolean algebra $B$.

%%$(B,\v,\w,{}^-,0,1)$.

Let $T(B)=\{(x_1,\ldots,x_n)\in B^{n-1}\mid x_1\leq\ldots\leq x_{n-1}\}$. On the set $T(B)$, we define an $\mbox{LM}_n$-algebra structure as follows:

\begin{itemize}
\item the lattice operations, as well as $0$ and $1$, are defined component--wise from $\Ld$;

\item for each $(x_1,\ldots,x_{n-1})\in T(B)$ and $i\in\{1,\ldots,n-1\}$ one has:\\
$N(x_1,\ldots x_{n-1})=(\ov{x_{n-1}},\ldots,\ov{x_1})$ and $\phii(x_1,\ldots,x_n)=(x_i,\ldots,x_i) .$
\end{itemize}
\end{exe}

\subsection{Generalized logic spaces defined by $LM_n$ algebraic logics}

\begin{itemize}
\item Topological semigroup spaces of topological automata
\item Topological groupoid spaces of reset automata modules 
\end{itemize}

\subsection{Axioms defining a generalized topos}
\begin{itemize}
\item Consider a subobject logic classifier $O$ defined as an LM-algebraic logic
$L_n$ in the category {\bf L} of LM-logic algebras, together with logic-valued functors $F_o: L \to V$, 
where $V$ is the class of N logic values, with $N$ needing not be finite. 
\item A triple $(O,{\bf L},F_o)$ defines a generalized topos, $\tau$, if the above axioms 
defining $O$ are satisfied, and if the functor $Fo$ is an univalued functor in the sense of Mitchell.
\end{itemize}

{\bf More to come...}


\subsection{Applications of generalized topoi:}
\begin{itemize}
\item Modern quantum logic (MQL)
\item Generalized quantum automata 
\item Mathematical models of N-state genetic networks \cite{BBGG1}
\item Mathematical models of parallel computing networks
\end{itemize}

\begin{thebibliography}{9}

\bibitem{GG-CV70}
Georgescu, G. and C. Vraciu. 1970, On the characterization of centered \L{}ukasiewicz
algebras., {\em J. Algebra}, \textbf{16}: 486-495.

\bibitem{GG2k6}
Georgescu, G. 2006, N-valued Logics and \L ukasiewicz-Moisil Algebras, \emph{Axiomathes}, \textbf{16} (1-2): 123-136.

\bibitem{ICB77}
Baianu, I.C.: 1977, A Logical Model of Genetic Activities in \L ukasiewicz Algebras: The Non-linear Theory. \emph{Bulletin of Mathematical Biology}, \textbf{39}: 249-258.

\bibitem{ICB2004a}
Baianu, I.C.: 2004a. \L{}ukasiewicz-Topos Models of Neural Networks, Cell Genome and Interactome Nonlinear Dynamic Models (2004). Eprint. Cogprints--Sussex Univ. 

\bibitem{ICB04b}
Baianu, I.C.: 2004b \L{}ukasiewicz-Topos Models of Neural Networks, Cell Genome and Interactome Nonlinear Dynamics). CERN Preprint EXT-2004-059. \textit{Health Physics and Radiation Effects} (June 29, 2004). 
 
\bibitem{Bgg2}
Baianu, I. C., Glazebrook, J. F. and G. Georgescu: 2004, Categories of Quantum Automata and N-Valued \L ukasiewicz Algebras in Relation to Dynamic Bionetworks, \textbf{(M,R)}--Systems and Their Higher Dimensional Algebra,
\PMlinkexternal{Abstract and Preprint of Report in PDF}{http://en.wikipedia.org/wiki/User:Bci2/Books/Interactomics} . 

\bibitem{BBGG1}
Baianu I. C., Brown R., Georgescu G. and J. F. Glazebrook: 2006b, Complex Nonlinear Biodynamics in Categories, Higher Dimensional Algebra and \L{}ukasiewicz--Moisil Topos: Transformations of Neuronal, Genetic and Neoplastic Networks., \emph{Axiomathes}, \textbf{16} Nos. 1--2: 65--122.

\bibitem{MB68}
Mitchell, Barry. The Theory of Categories. Academic Press: London, 1968.

\end{thebibliography}


%%%%%
%%%%%
\end{document}
