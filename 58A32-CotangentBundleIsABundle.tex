\documentclass[12pt]{article}
\usepackage{pmmeta}
\pmcanonicalname{CotangentBundleIsABundle}
\pmcreated{2013-03-22 14:54:31}
\pmmodified{2013-03-22 14:54:31}
\pmowner{rspuzio}{6075}
\pmmodifier{rspuzio}{6075}
\pmtitle{cotangent bundle is a bundle}
\pmrecord{6}{36593}
\pmprivacy{1}
\pmauthor{rspuzio}{6075}
\pmtype{Proof}
\pmcomment{trigger rebuild}
\pmclassification{msc}{58A32}

\endmetadata

% this is the default PlanetMath preamble.  as your knowledge
% of TeX increases, you will probably want to edit this, but
% it should be fine as is for beginners.

% almost certainly you want these
\usepackage{amssymb}
\usepackage{amsmath}
\usepackage{amsfonts}

% used for TeXing text within eps files
%\usepackage{psfrag}
% need this for including graphics (\includegraphics)
%\usepackage{graphicx}
% for neatly defining theorems and propositions
%\usepackage{amsthm}
% making logically defined graphics
%%%\usepackage{xypic}

% there are many more packages, add them here as you need them

% define commands here
\begin{document}
Verifying the first criterion is simply a matter of writing it out:
 $$(x^1, \ldots, x^{2n}) \mapsto ( \pi (x^1, \ldots, x^{2n}), \phi_\alpha (x^1, \ldots, x^{n})) = ( (x^1, \ldots, x^n), (x^{n+1}, \ldots, x^{2n}) )$$
This is obviously a homeomorphism.

As for the second criterion,
\begin{eqnarray*}
\sum_{j = 1}^n {g_{\alpha \beta}}^i_j (x^1, \ldots, x^{2n}) {\phi_\beta}^j (x^1, \ldots, x^{2n}) &=& \sum_{j = n}^{2n} {\partial \big( \sigma_{\alpha \beta} (x^1, \ldots x^n) \big)^i \over \partial x^j} x^{j+n} \\ &=& {{\sigma'}_{\alpha \beta}}^{i+n} (x^1, \ldots, x^{2n}) \\ &=& {\phi_\alpha}^i (x^1, \ldots, x^{2n})
\end{eqnarray*}

The third criterion follows from the chain rule:
\begin{eqnarray*}
{g_{\alpha \beta}}^i_j  {g_{\beta \gamma}}^j_k = {\partial \big( \sigma_{\alpha \beta} (x^1, \ldots x^n) \big)^i \over \partial x^j} {\partial \big( \sigma_{\beta \gamma} (x^1, \ldots x^n) \big)^j \over \partial x^k}
\end{eqnarray*}
%%%%%
%%%%%
\end{document}
